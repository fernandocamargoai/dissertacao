\documentclass{beamer}

\usepackage[latin1]{inputenc}
\usepackage[brazil]{babel}

\usetheme{Berlin}
\usecolortheme{default}

\usepackage{multicol}
\usepackage{amsmath}
\usepackage{colortbl}
%\usepackage[figuresleft]{rotating}
\usepackage{rotating}
%\usepackage[table*]{xcolor}
\xdefinecolor{gray95}{gray}{0.65}
\xdefinecolor{gray25}{gray}{0.8}

\graphicspath{{./fig/}}

% Colocando numero de paginas no slide
\setbeamertemplate{footline}[frame number]
% Desativando os botoes de navegacao
\beamertemplatenavigationsymbolsempty
% Layout da pagina
\hypersetup{pdfpagelayout=SinglePage}


%Information to be included in the title page:
\title{Aplica��o de Meta Heur�sticas na Otimiza��o Multiobjetivo de Sistemas Hidrot�rmicos}
\author[Camargo, F. H. F.]{Fernando Henrique Fernandes de Camargo}
\institute[EMC - UFG]{Escola de Engenharia El�trica, Mec�nica e de Computa��o\\Universidade Federal de Goi�s}
\date{\today}

\AtBeginSection[]
{
  \begin{frame}
    \frametitle{Tabela de conte�do}
    \begin{multicols}{2}
      \tableofcontents[currentsection]
    \end{multicols}
  \end{frame}
}

\begin{document}

\frame{\titlepage}

\begin{frame}
\frametitle{Tabela de conte�do}
\begin{multicols}{2}
  \tableofcontents
\end{multicols}
\end{frame}

\section[O Problema]{Planejamento Energ�tico de Sistemas Hidrot�rmicos}

\begin{frame}
  \frametitle{Teste 1}
  Testando o teste 1
  
  \begin{itemize}
    \item<1-> Um
    \item<2-> Dois
    \item<3-> Tr�s
    \item<4-> Quatro
  \end{itemize}
\end{frame}

\subsection{Sistemas Hidrot�rmicos de Gera��o}

\subsection{T�cnicas Utilizadas}

\subsection{Formula��o}

\section[Otimiza��o Multiobjetivo]{Otimiza��o Multiobjetivo}

\subsection{Fronteira de Pareto}

\subsection{M�tricas de Qualidade}

\section[Algoritmos Aplicados]{Algoritmos Aplicados}

\subsection{NSGA-II}

\subsection{SPEA2}

\subsection{MOCell}

\subsection{OMOPSO}

\subsection{SMPSO}

\section[Estudos de Caso]{Estudos de Caso}

\subsection{Cen�rio de Teste}

\subsection{Arquitetura do Sistema}

\subsection{Par�metros Utilizados nos Algoritmos}

\section[Resultados]{Resultados}

\subsection{Per�odo 1951-1956}

\subsection{Per�odo 1961-1966}

\subsection{Per�odo 1980-1985}

\subsection{Resultados Finais}

\section[Conclus�es]{Conclus�es e Trabalhos Futuros}




\begin{frame}
  \frametitle{Teste 1}
  Testando o teste 1
  
  \begin{itemize}
    \item<1-> Um
    \item<2-> Dois
    \item<3-> Tr�s
    \item<4-> Quatro
  \end{itemize}
\end{frame}

\end{document}
